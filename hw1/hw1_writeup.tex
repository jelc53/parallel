\documentclass[12pt,letterpaper,twoside]{article}

\newif\ifsolution\solutiontrue   % Include the solutions
%\newif\ifsolution\solutionfalse  % Exclude the solutions

\usepackage{cme213}
\usepackage{xcolor}

\newcommand{\T}[1]{\text{\texttt{#1}}}
\newcommand{\V}[1]{\text{\textit{#1}}}

\begin{document}

{\centering \textbf{Homework 1\\ Due Friday, April 8th via GradeScope\\}}
\vspace*{-8pt}\noindent\rule{\linewidth}{1pt}

\paragraph{Question 0: } Our graph is undirected, but we could use it to build a
\href{https://en.wikipedia.org/wiki/Directed_graph#Definition}{\emph{directed} graph} or \emph{digraph}. 
A digraph's edges are \emph{ordered} pairs of nodes $(n_i, n_j)$, rather than unordered pairs $\{n_i, n_j\}$; 
all operations look like their undirected counterparts and have similar complexity.

Here's one digraph representation using our graph class.
\begin{cpp}
template <typename V> 
class Digraph { 
... 
private:
Graph<V> g_;
};
\end{cpp}
Explain the representation by giving an abstraction function and
representation invariant. Note that \texttt{Digraph} is not
\texttt{Graph}'s friend and can only use \texttt{Graph}'s public
interface. \\

{\color{purple}Idea: Want to represent each node of the directed graph with two nodes of 
the undirected graph , i.e. an output (even idx) and input (odd idx).

$AF$(Digraph) = ($N$, $E$) where
\begin{itemize}
    \item $N = \{\{n_{2i}, n_{2i+1}\}$ $|$ $0 <= i <$ g\_.num\_nodes()\}
    \item $E = \{\{n_{i}, n_{j}\}$ $|$ j - i $\%$ 2 == 1, i $\in$ adj\_[j]\}
\end{itemize}

$RI$(Digraph): 
\begin{itemize}
    \item Size of nodes set $N$ = 2 * g\_.num\_nodes()
    \item Size of edges set $E$ = g\_.num\_edges()
\end{itemize}

}


\paragraph{Question 1: } The \href{https://en.cppreference.com/w/cpp/container/vector/resize}{\texttt{vector<T>::resize}} 
operation extends a vector with \emph{memory initialized} elements. (For example, after \texttt{vector<int> v; v.resize(20)}, \texttt{v} 
contains 20 zeroes in memory.) This is almost always what we want, but it can be useful to extend a vector with \emph{uninitialized
  memory}. Here's an example: uninitialized memory helps build a \emph{sparse
  vector}, in which elements are initialized only when first referenced. We
assume a special \texttt{garbage\_vector} type whose \texttt{resize} method does
not initialize new memory. \textbf{See next page}\\ \textbf{Note:} This question will
not completely make sense until you understand what the code is doing below!

\newpage
\begin{cpp}
template <typename T> 
class sparse_vector {
garbage_vector<size_t> position_;
vector<size_t> check_;
vector<T> value_;
public:
// Construct a sparse vector with all elements equal to T().
sparse_vector() {}
// Return a reference to the element at position @a i.
T& operator[](size_t i) {
    // If out of bounds, we must re-size our array and add 
    // (uninitialized) memory!
    if (i >= position_.size())
    position_.resize(i + 1);
    // If we haven't initialized the memory yet, go ahead and do so now.
    if (position_[i] >= check_.size() || check_[position_[i]] != i) {
    position_[i] = check_.size(); // First reference to the element ati.
    check_.push_back(i);          // We associate an index...
    value_.push_back(T());        // ...alongside a value.
    }
    return value_[position_[i]];
}
};
\end{cpp}

An abstract \texttt{sparse\_vector} value is an \textbf{infinite} vector.

Write an abstraction function and representation invariant for
\texttt{sparse\_vector}.

{\color{purple}Idea: Want to describe the functionality and representation invariants
of our infinite vector. In particular, how does the implementation above guarantee operator 
[] works for all positive integer inputs i without initializing memory address of locations
for each resize. 

$AF$(sparse\_vector) = an infinite vector $V$ such that $V[i]$ returns an integer 
for all $i \in $ positive integers

$RI$(sparse\_vector): 
\begin{itemize}
    \item check\_.size() == value\_.size()
    \item *max\_element(check\_.begin(), check\_.end()) + 1 = position\_.size()
\end{itemize}

}



\end{document}
