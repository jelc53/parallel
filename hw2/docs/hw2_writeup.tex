\documentclass[12pt,letterpaper,twoside]{article}

\newif\ifsolution\solutiontrue   % Include the solutions
%\newif\ifsolution\solutionfalse  % Exclude the solutions

\usepackage{cme213}
\usepackage{xcolor}

\newcommand{\T}[1]{\text{\texttt{#1}}}
\newcommand{\V}[1]{\text{\textit{#1}}}

\begin{document}

{\centering \textbf{Homework 2\\ Due Friday, April 22nd via GradeScope\\}}
\vspace*{-8pt}\noindent\rule{\linewidth}{1pt}

\paragraph{Problem 1: } Implement a parallel function that sums separately the odd
and even values of a vector. 

Idea: Need to implement \texttt{parallelSum} using \texttt{omp parallel for} with 
reductions for both even and odd accumulators.

\begin{cpp}
std::vector<uint> parallelSum(const std::vector<uint> &v) 
{
    omp_set_num_threads(4);
    std::vector<uint> sums(2);
    uint sum0 = 0; uint sum1 = 0;

    #pragma omp parallel for reduction(+:sum0) reduction(+:sum1)
    for(uint i=0; i<v.size(); i++) {
        if (v[i] % 2 == 0) {
            sum0 += v[i];
        }
        else {
            sum1 += v[i];
        }
    }
    sums[0] = sum0; sums[1] = sum1;
    return sums;
}
\end{cpp}

Console logs from \texttt{main\_q1.cpp}.
\begin{verbatim}
$ make main_q1
g++ -std=c++11 -g -Wall -O3 -fopenmp main_q1.cpp -o main_q1

$ ./main_q1
Parallel
Sum Even: 757361650
Sum Odd: 742539102
Time: 0.00433168
Serial
Sum Even: 757361650
Sum Odd: 742539102
Time: 0.106256
main_q1.cpp:60:main     TEST PASSED.
\end{verbatim}


\paragraph{Problem 2: } Implement Radix Sort in parallel ...

\begin{itemize}
    \item \textbf{Question 1: computeBlockHistograms()} Idea: using \texttt{openMP}
    to parallelize computation across "blocks" when creaing local histograms.
    Code must pass Test1().

    \begin{verbatim}
    $ make main_q2
    g++ -std=c++11 -g -Wall -O3 -fopenmp main_q2.cpp -o main_q2

    $ ./main_q2
    tests_q2.h:22:Test1     TEST PASSED.
    \end{verbatim}

    \item \textbf{Question 2: reduceLocalHistoToGlobal()} Idea: accumulate values 
    based on the remainder of the \texttt{index} divided by \texttt{bucketSize}.
    Code must pass Test2().

    \begin{verbatim}
    $ make main_q2
    g++ -std=c++11 -g -Wall -O3 -fopenmp main_q2.cpp -o main_q2

    $ ./main_q2
    tests_q2.h:38:Test2     TEST PASSED.
    \end{verbatim}

    \item \textbf{Question 3: scanGlobalHisto()} Idea: implement cumulative sum 
    using \texttt{std::partial\_sum} standard library function. Note, needed to 
    adjust \texttt{Output Iterator} and \texttt{Input Iterator} to ensure we 
    begin at zero and do not inadvertedly overflow.

    \begin{verbatim}
    $ make main_q2
    g++ -std=c++11 -g -Wall -O3 -fopenmp main_q2.cpp -o main_q2

    $ ./main_q2
    tests_q2.h:50:Test3     TEST PASSED.
    \end{verbatim}

    \item \textbf{Question 4: computeBlockExScanFromGlobalHisto()} Idea: populate first
    using \texttt{globalHistoScan} and then increment using \texttt{blockHisograms}
    for subseuqent blocks. This has the effect of splitting the global histogram
    into blocks need to update our sorting algorithm (next step).
    
    \begin{verbatim}
    $ make main_q2
    g++ -std=c++11 -g -Wall -O3 -fopenmp main_q2.cpp -o main_q2

    $ ./main_q2
    tests_q2.h:67:Test4     TEST PASSED.
    \end{verbatim}

    \item \textbf{Question 5: populateOutputFromBlockExScan()} Idea: use pre-computed
    \texttt{blockEx Scan} to help target where entries of our unsorted input vector
    should map to in \texttt{sorted}. We can parallelize this operation by block using 
    \texttt{openMP}. Note, we still need to re-compute which "bucket" each of our unsorted 
    entries are from at each step since this information is not stored and passed to the 
    function. 
    
    Also, this step only succeeds in sorting our input up to the \texttt{numBits}'th 
    bit (in this case 8 bits of sorting per pass). Subsequent "passes" are needed to complete
    our radix sort algorithm since many input entires are greater than 256 (limit of 8 bits).

    \begin{verbatim}
    $ make main_q2
    g++ -std=c++11 -g -Wall -O3 -fopenmp main_q2.cpp -o main_q2

    $ ./main_q2
    tests_q2.h:84:Test5     TEST PASSED.
    \end{verbatim}

\end{itemize}

Submission information logs.
\begin{verbatim}
$ /afs/ir.stanford.edu/class/cme213/script/submit.py hw1 private/cme213-jelc53/hw1
Submission for assignment 'hw1' as user 'jelc'
Attempt 1/10
Time stamp: 2022-04-01 20:53
List of files being copied:
    private/cme213-jelc53/hw1/main_q1.cpp	 [3875 bytes]
    private/cme213-jelc53/hw1/main_q2.cpp	 [1213 bytes]
    private/cme213-jelc53/hw1/main_q3.cpp	 [1362 bytes]
    private/cme213-jelc53/hw1/main_q4.cpp	 [5117 bytes]
    private/cme213-jelc53/hw1/matrix.hpp	 [3036 bytes]

Your files were copied successfully.
Directory where files were copied: /afs/ir.stanford.edu/class/cme213/submissions/hw1/jelc/1
List of files in this directory:
    main_q1.cpp	 [3875 bytes]
    main_q2.cpp	 [1213 bytes]
    main_q3.cpp	 [1362 bytes]
    main_q4.cpp	 [5117 bytes]
    matrix.hpp	 [3036 bytes]
    metadata	 [137 bytes]

This completes the submission process. Thank you!
\end{verbatim}

\end{document}
