\documentclass[12pt,letterpaper,twoside]{article}

\newif\ifsolution\solutiontrue   % Include the solutions
%\newif\ifsolution\solutionfalse  % Exclude the solutions

\usepackage{cme213}
\usepackage{xcolor}

\newcommand{\T}[1]{\text{\texttt{#1}}}
\newcommand{\V}[1]{\text{\textit{#1}}}

\begin{document}

{\centering \textbf{Homework 3\\ Due Sunday, May 1st via GradeScope\\}}
\vspace*{-8pt}\noindent\rule{\linewidth}{1pt}

\paragraph{Problem 1: } Implement a parallel function that sums separately the odd
and even values of a vector. 

Idea: Need to implement \texttt{parallelSum} using \texttt{omp parallel for} with 
reductions for both even and odd accumulators.

\begin{cpp}
std::vector<uint> parallelSum(const std::vector<uint> &v) 
{
    omp_set_num_threads(4);
    std::vector<uint> sums(2);
    uint sum0 = 0; uint sum1 = 0;

    #pragma omp parallel for reduction(+:sum0) reduction(+:sum1)
    for(uint i=0; i<v.size(); i++) {
        if (v[i] % 2 == 0) {
            sum0 += v[i];
        }
        else {
            sum1 += v[i];
        }
    }
    sums[0] = sum0; sums[1] = sum1;
    return sums;
}
\end{cpp}

Console logs from \texttt{main\_q1.cpp}.
\begin{verbatim}
$ make main_q1
g++ -std=c++11 -g -Wall -O3 -fopenmp main_q1.cpp -o main_q1

$ ./main_q1
Parallel
Sum Even: 757361650
Sum Odd: 742539102
Time: 0.00433168
Serial
Sum Even: 757361650
Sum Odd: 742539102
Time: 0.106256
main_q1.cpp:60:main     TEST PASSED.
\end{verbatim}


\paragraph{Problem 2: } Implement Radix Sort algorithm in parallel.

\begin{itemize}
    \item \textbf{Question 1: computeBlockHistograms()} Idea: using \texttt{openMP}
    to parallelize computation across "blocks" when creaing local histograms.
    Code must pass Test1().

    \begin{verbatim}
    $ make main_q2
    g++ -std=c++11 -g -Wall -O3 -fopenmp main_q2.cpp -o main_q2

    $ ./main_q2
    tests_q2.h:22:Test1     TEST PASSED.
    \end{verbatim}

    \item \textbf{Question 2: reduceLocalHistoToGlobal()} Idea: accumulate values 
    based on the remainder of the \texttt{index} divided by \texttt{bucketSize}.
    Code must pass Test2().

    \begin{verbatim}
    $ make main_q2
    g++ -std=c++11 -g -Wall -O3 -fopenmp main_q2.cpp -o main_q2

    $ ./main_q2
    tests_q2.h:38:Test2     TEST PASSED.
    \end{verbatim}

    \item \textbf{Question 3: scanGlobalHisto()} Idea: implement cumulative sum 
    using \texttt{std::partial\_sum} standard library function. Note, needed to 
    adjust \texttt{Output Iterator} and \texttt{Input Iterator} to ensure we 
    begin at zero and do not inadvertedly overflow.

    \begin{verbatim}
    $ make main_q2
    g++ -std=c++11 -g -Wall -O3 -fopenmp main_q2.cpp -o main_q2

    $ ./main_q2
    tests_q2.h:50:Test3     TEST PASSED.
    \end{verbatim}

    \item \textbf{Question 4: computeBlockExScanFromGlobalHisto()} Idea: populate first
    using \texttt{globalHistoScan} and then increment using \texttt{blockHisograms}
    for subseuqent blocks. This has the effect of splitting the global histogram
    into blocks need to update our sorting algorithm (next step).
    
    \begin{verbatim}
    $ make main_q2
    g++ -std=c++11 -g -Wall -O3 -fopenmp main_q2.cpp -o main_q2

    $ ./main_q2
    tests_q2.h:67:Test4     TEST PASSED.
    \end{verbatim}

    \item \textbf{Question 5: populateOutputFromBlockExScan()} Idea: use pre-computed
    \texttt{blockEx Scan} to help target where entries of our unsorted input vector
    should map to in \texttt{sorted}. We can parallelize this operation by block using 
    \texttt{openMP}. Note, we still need to re-compute which "bucket" each of our unsorted 
    entries are from at each step since this information is not stored and passed to the 
    function. 
    
    Also, this step only succeeds in sorting our input up to the \texttt{numBits}'th 
    bit (in this case 8 bits of sorting per pass). Subsequent "passes" are needed to complete
    our radix sort algorithm since many input entires are greater than 256 (limit of 8 bits).

    \begin{verbatim}
    $ make main_q2
    g++ -std=c++11 -g -Wall -O3 -fopenmp main_q2.cpp -o main_q2

    $ ./main_q2
    tests_q2.h:84:Test5     TEST PASSED.
    \end{verbatim}

    \item \textbf{Question 6: Serial vs parallel benchmarking} Use ICME GPU cluster to 
    compare time estimates for different numbers of threads and blocks. In order to get 
    this coded to run I needed to handle exception case where the number of blocks did 
    not cleanly divide the number of elements in keys. In these cases, the derived block 
    size implied we would not sort the end of the keys list. My solution was to pad the 
    keys list with zeros until it evenly divided the next block number and remove this 
    number of zero elements after sorting.

\begin{verbatim}
jelc@icme-gpu:~/cme213-para/hw2$ make
g++ -std=c++11 -g -Wall -O3 -fopenmp main_q1.cpp -o main_q1
g++ -std=c++11 -g -Wall -O3 -fopenmp main_q2.cpp -o main_q2
g++ -std=c++11 -g -Wall -O3 -fopenmp main_q2.cpp -D QUESTION6 -o main_q2_part6

jelc@icme-gpu:~/cme213-para/hw2$ sbatch script.sh 
Submitted batch job 41096

jelc@icme-gpu:~/cme213-para/hw2$ cat job_41096.out 
Date				= Wed Apr 20 06:31:24 UTC 2022
Hostname				= icmet01
Working directory 		= /home/jelc/cme213-para/hw2

Number of nodes allocated		= 1
Number of cores/task allocated 	= 16
tests_q2.h:22:Test1	TEST PASSED.
tests_q2.h:38:Test2	TEST PASSED.
tests_q2.h:50:Test3	TEST PASSED.
tests_q2.h:67:Test4	TEST PASSED.
tests_q2.h:84:Test5	TEST PASSED.
Serial Radix Sort: PASS
Parallel Radix Sort: PASS
stl: 0.310788
serial radix: 0.0461267
parallel radix: 0.031369
Threads Blocks / Timing
         1       2       4       8      12      16      24      32      40      48
   1    0.058   0.052   0.046   0.048   0.059   0.056   0.059   0.068   0.063   0.069
   2    0.045   0.033   0.038   0.035   0.036   0.043   0.042   0.041   0.057   0.047
   4    0.050   0.034   0.022   0.022   0.028   0.025   0.030   0.030   0.036   0.036
   8    0.057   0.031   0.021   0.015   0.019   0.018   0.022   0.022   0.027   0.029
  12    0.069   0.035   0.026   0.020   0.020   0.024   0.024   0.025   0.028   0.031
  16    0.070   0.033   0.027   0.020   0.019   0.019   0.026   0.024   0.030   0.031
  24    0.065   0.043   0.028   0.027   0.023   0.022   0.035   0.031   0.034   0.038
  32    0.066   0.041   0.034   0.031   0.023   0.023   0.029   0.029   0.033   0.034
  40    0.064   0.042   0.038   0.029   0.024   0.028   0.030   0.029   0.031   0.036
  48    0.065   0.036   0.028   0.029   0.025   0.026   0.031   0.029   0.032   0.040
Benchmark runs: PASS
\end{verbatim} 
    
    Optimal time of 0.015 seconds achieved at 8 threads / 8 blocks on the icme gpu cluster.
    See benchmark results above.

    In general, our performance improves as we increase number of threads up available to 
    compute our sorting program in parallel. However, there is a trade-off between extra 
    parallelization and overhead required to generate and manage threads. For the scenarios
    we ran, the best performance was typically around 8-16 threads. Also since the icme gpu 
    cluster allocates a maximum of 16 cpu cores per task, we should expect to hit
    a hardware limit at 32 threads (2x threads per core). 
    
    Similarly, for number of blocks the optimal is driven by a tradeoff between increased 
    ability to parallelize sorting operations and overhead associated with combining sorted 
    blocks. For the scenarios we ran, the best performance was typically around 8-16 blocks.

\end{itemize}


Submission information logs.
\begin{verbatim}
jelc@cardinal1:~$ /afs/ir.stanford.edu/class/cme213/script/submit.py hw2 private/cme213-jelc53/hw2
Submission for assignment 'hw2' as user 'jelc'
Attempt 1/10
Time stamp: 2022-04-13 20:09
List of files being copied:
    private/cme213-jelc53/hw2/sum.h	 [768 bytes]
    private/cme213-jelc53/hw2/parallel_radix_sort.h	 [7625 bytes]

Your files were copied successfully.
Directory where files were copied: /afs/ir.stanford.edu/class/cme213/submissions/hw2/jelc/1
List of files in this directory:
    sum.h	 [768 bytes]
    parallel_radix_sort.h	 [7625 bytes]
    metadata	 [137 bytes]

This completes the submission process. Thank you!

jelc@cardinal1:~$ ls /afs/ir.stanford.edu/class/cme213/submissions/hw2/jelc/1
metadata  parallel_radix_sort.h  sum.h
\end{verbatim}

\end{document}
